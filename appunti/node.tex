\documentclass[a4paper,11pt,twoside]{article}
\usepackage[a4paper, total={7in, 10in}]{geometry} %for manage the page geometry

\usepackage{titlesec}%for titleformat
\usepackage{titling}
\usepackage[english,italian]{babel}
\usepackage{hyperref}

\pagestyle{plain} %insert number of page
\titleformat{\section}
{\bfseries } %argument for formatting title section
{}
{0em} %distance between number and title of the section
{}  %space from border to the title


\titleformat{\section}
{\bfseries\LARGE}
{}
{0em}
{}

\titleformat{\subsection}
{\bfseries\LARGE}
{}
{0em}
{}

\titleformat{\subsubsection}[runin]
{\large}
{}
{0em}
{}

\titlespacing{\subsubsection}
{0em}   %space of the margin
{0em}
{0em}

%redefine maketitle
\renewcommand{\maketitle}
{
    \begin{center}
        \huge\bfseries
        \thetitle
        \\ {}
        \vspace{.0em} %vertical space
    \end{center}
}

\begin{document}
    \title{Node}
    \author{Nico Fiorini}
    \maketitle

    \section{Introduzione}

    Cos'è Node ? \\{}
    Node js is an open source and cross-platform runtime envrironment for 
    executing java script code outside of a browser.\\{}
    Quite often we use node to obuild back-end services also called API (Application Programming Interface)
    these are the services that power our client applications, like a web app running insede 
    of a web browser or mobile app running on a mobile device, these client apps are 
    simply what user sees and interacts with they're, just surface. 

    
    Node is ideal for building higly-scalable data intensive and real-time back-end services
    that power our client apps.

    Node is easy to get started and can be used for prototyping and agile development.
    

    \section{Node architecture}
    Before Node we use javascript only to build application that run inside of a browser
    , so every browser out there has what we call a javascript engine. 


    That takes javascript code, and transforms it that computer can understand\vspace{1cm}  

    
    \textbf{Example} \\{}
    --microsoft edge uses chacra \\{}
    --firefox uses SpiderMonkey \\{}
    --chrome uses v8\\{}

    
    \hspace{1cm}

    for these varieties of engines that javascript code can behave in different way
    in one browser or another.

    
    
    In 2009\textit{Ryan Dahl} the creator of Node came up with brilliant idea
    he thought it would be great to execute javascript outside of a browser.


    He took v8, and embedded it inside a c++ program and called that program node. We have 
    a different object from the environment objects we have in browser.

\vspace{1cm}
    \textbf{Example} \\{}
    We don't have the document object  ---\(>\)document.getElementById(''); \href{https://en.wikipedia.org/wiki/Document_Object_Model}{See on wikipedia} 
    
\vspace{1cm}
    But i can work with file system -- fs.readFile()\\{}
    listen for requests --http.createServer()  and a given port and so on.


    Inside of a browser we can't do that.


    In essence: Node is a program that includes the v8 javascript engine plus and additional modules
    that give us capabilities not available inside browser. 
    browser and node have the same engine, but have different runtime environment for javascript.


    Node is not a programming language, and isn't frameworks for building web application, 
    it's a runtime environment for executing javascript.



   \newpage
   \section{How node works}
   Node is higly-scalable, this is because of the non-blocking or asynchoronous natre of 
   of node.Node applications are asynchronous by default.


   Node is ideal for I/O-intensive apps
   %\textbf{What do you mean asynchoronous?}
   
   I continue in the folder /home/linic/node 


   \section{Node Module System}
   What modules are? 


   Why we need them and how they work?


   we'll explore a feel of the modules built into the core of node such as operating system, file system, events and http. 

   \section{Global object}

   //continue in Node folder with index2.js

    \section{Modules}

    In the client-side javascript that we run inside of browser, when we declare a variable or a function 
    that is addet to the Global scope for. 
    continue in index3.js


    So rimasto a 22'



    



    
\end{document}

